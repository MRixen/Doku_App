\usepackage{filecontents}
\begin{filecontents}{appendixtoc.sty}
%
% appendixtoc.sty
% Copyright (c) Markus Kohm, 2013-2014
% See `appendixtocexample.tex' for license informations. Distribution without
% `appendixtocexample.tex' is forbidden!
% See <http://www.komascript.de/comment/3447#comment-3447> for more information.
\ProvidesPackage{appendixtoc}[2014/01/22 unsupported LaTeX2e package]
\RequirePackage{scrbase}[2013/12/19]% frühere Versionen unterstützen keine Sprachliste bei \providecaptionname
\RequirePackage{tocstyle}
\usetocstyle{KOMAlike}
% Die folgende Umgebung wird verwendet, um innerhalb der toc-Datei einzelne
% Bereiche ein- und ausschalten zu können. In die toc-Datei wird die Umgebung
% dabei jeweils als \begin{tocconditional}{BEREICH}...\end{tocconditional}
% eingefügt.
\newenvironment*{tocconditional}[1]{%
  \expandafter\ifx\csname if@toccond@#1\expandafter\endcsname
                  \csname iftrue\endcsname
  \else
    \value{tocdepth}=-10000\relax
  \fi
  \typeout{tocdepth in `#1': \the\c@tocdepth}%
}{%
}
 
% Gleich nach dem Öffnen der toc-Datei beginnen wir den Haupt-Bereich "main":
\AtBeginDocument{%
  \addtocontents{toc}{\string\begin{tocconditional}{main}}
}
% Und der letzte Bereich endet am Ende der toc-Datei.
\BeforeClosingMainAux{%
  \addtocontents{toc}{\string\end{tocconditional}}%
}
 
% Hier können nun neue Bereiche definiert (wie man das
% macht zeigen wir gleich im Anschluss) ...
\newcommand*{\newtocconditional}[2][false]{%
  \expandafter\newif\csname if@toccond@#2\endcsname
  \csname @toccond@#2#1\endcsname
}
% ... und ein- oder ausgeschaltet werden.
% (Beispiele für die Verwendung von \settocconditional sind
% weiter unten bei der Definition von \appendixtableofcontents
% zu finden.)
\newcommand*{\settocconditional}[2]{%
  \csname @toccond@#1#2\endcsname
}
 
% Neben dem (bereits aktivierten) Hauptbereich ...
\newtocconditional[true]{main}
% ... definieren wir noch einen (noch nicht aktivierten)
% Bereich für den Anhang.
\newtocconditional{appendix}
 
% Mit dem Anhang geben wir einerseits das Anhangsverzeichnis aus,
% andererseits beenden wir den aktuellen Bereich in der toc-Datei und beginnen
% den neuen Bereich "appendix". Damit im Haupt-Inhaltsverzeichnis ein Eintrag
% für das Anhangsverzeichnis erscheint, verwenden wir \addchap und zwar noch
% bevor der letzte Bereich geschlossen wird. Wenn wir es ganz sicher machen
% wollten, müssten wir die auskommentierten Zeilen noch aktivieren. So
% verlassen wir uns einfach darauf, dass vor dem appendix-Bereich der
% main-Bereich lag.
\g@addto@macro\appendix{%
%  \addtocontents{toc}{\string\end{tocconditional}^^J
%    \string\begin{tocconditional}{main}}%
  \begingroup
    \@ifundefined{tocbasic@listhead}{% Falls \tocbasic@listhead (wird von
                               % KOMA-Script-Klassen verwendet) nicht
                               % definiert ist
      \@ifundefined{chapter}{% und falls \chapter nicht definiert ist,
        \section*{\listofappendixname}% \section* verwenden
      }{% aber falls \chapter definiert ist,
        \chapter*{\listofappendixname}% \chapter* verwenden
      }%
      % und noch die Kolumnentitel passend setzen.
      \@mkboth{\csname MakeMarkcase\endcsname{\listofappendixname}}%
              {\csname MakeMarkcase\endcsname{\listofappendixname}}%
    }{% Falls \toc@heading definiert ist,
      \def\@currext{appendix}% initialisieren
      \tocbasic@listhead{\listofappendixname}% und verwenden
    }%
  \endgroup
  \addtocontents{toc}{\string\end{tocconditional}^^J
    \string\begin{tocconditional}{appendix}}%
  \appendixtableofcontents
}
 
% Jetzt definieren wir das Anhangsverzeichnis selbst als Alias für die
% toc-Datei. Dabei wird aber der Hauptbereich "main" deaktiviert und der
% Anhangsbereich "appendix" aktiviert.
\newcommand*{\appendixtableofcontents}{%
  \showtoc[{ %
    \aliastoc{\tocstyleTOC}{toc}%
    \settocconditional{main}{false}%
    \settocconditional{appendix}{true}%
  }]{toc}%
}
 
% Auch wenn man einen Anhang normalerweise nicht beenden kann, so ist es
% ggf. erwünscht, dass Literaturverzeichnis, Index etc. zwar nach den Kapiteln
% des Anhangs kommen, aber dem Hauptverzeichnis zugeordnet werden sollen. Also
% benötigen wir eine Anweisung, um in der toc-Datei den aktuellen Bereich zu
% beenden und wieder einen Hauptbereich einzuschalten:
\newcommand*{\postappendix}{%
  \addtocontents{toc}{\string\end{tocconditional}^^J%
      \string\begin{tocconditional}{main}}%
}
 
% Den Namen definieren:
\newcommand*{\listofappendixname}{Table of appendices}
\AtBeginDocument{%
  \providecaptionname{american,australien,british,canadian,english,UKenglish,USenglish}\listofappendixname{Table of appendices}%
  \providecaptionname{german,ngerman,austrian,naustrian,swissgerman,nswissgerman}\listofappendixname{Anhangsverzeichnis}%
}%
\end{filecontents}
 
\usepackage{appendixtoc}
% Wir wollen das Anhangsverzeichnis im Inhaltsverzeichnis, also sorgen wir
% dafür, dass das Paket tocbasic geladen ist (auch, wenn keine
% KOMA-Script-Klasse verwendet wird). Das muss unbedingt _vor_ dem Laden von
% appendixtoc passieren!
\usepackage{tocbasic}
\usepackage{appendixtoc}
\setuptoc{appendix}{totoc}% dank tocbasic geht das jetzt so einfach