\chapter{Einleitung}
%
% Allgmein
%
Die \gls{app} \textit{Fast Diagnose} ist dazu bestimmt, über ein laufendes 
System mit einer ABB-Robotersteuerung Informationen abzurufen. Diese Daten 
enthalten zunächst die tabellarisch zusammengefassten Projektdaten (SOP, FAT, 
Betriebsstunden, Projekt-Nr., etc.) zudem wird die Zykluszeit als aktueller, 
sowie als Mittelwert ausgegeben. Neben den Logs wird der Meldungstext von 
System-Events (Information, Warnung, Fehler) ebenfalls dargestellt.\\
%
% Ausgangspunkt
%
Bisher konnten Systeminformationen ausschließlich über das FlexPendant der 
ABB-Steuerung ausgegeben werden. Die App Fast Diagnose erleichtert die 
Inbetriebnahme der Anlage beim Arbeitgeber und Arbeitnehmer, denn wichtige 
Systeminformationen werden direkt auf dem Smartphone dargestellt. Wenn im 
Testbetrieb die ANlage stehen bleibt, so wird der Inbetriebnehmer duch das 
Vibrieren des Smartphones und dem Darstellen eines Dialogs darauf aufmerkam 
gemacht. Es ist somit nicht zwingend direkt an der Anlage anwesend zu sein.
Zudem kann bei FAT und SOP schnell und bequem die Takzeit kontrolliert werden.\\
%
% Kapitelinhalt
%
Das Kapitel \ref{sec:Nachrichtenaufbau} befasst sich mit dem Aufbau der 
Nachricht und \ref{sec:Programmaufbau} zeigt dden PRogrammaufbau, d.h. die 
Tasks, Module und Klassen. In \ref{sec:HMI} werden die grafischen Elemente 
erläutert. Kapitel \ref{sec:Ausblick} beinhaltet eine offene Punkte Liste.