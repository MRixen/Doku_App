\subsection{ABB}
\label{sec:ChapterABB}
Das System besteht aus mehreren Tasks von denen vier für die Kommunikation 
zwischen dem Abb-Controller und der Android-\gls{app} fungieren.
Die jeweilige Kernaufgabe der Tasks wird im Folgenden erläutert und die 
wichtigsten Code-Abschnitte aufgezeigt. \\
Die Buffergröße beläuft sich aktuell auf 25 Einträge.

\subsubsection{Task ServerComm}
Der Task \textit{ServerComm} beinhaltet das Modul \textit{ServerComm} und ist 
für die 
grundsätzliche Kommunikation, d.h. für 
das Senden und 
Empfangen von Daten, zuständig.
In den Zeilen 1 - 5 des Listings \ref{abbCode:moduleServerComm} ist die 
Hauptroutine zu sehen, in welcher der Routinen-Aufruf \textit{waitForClients()} 
endlos erfolgt.
In dieser Routine wird der Socket von Case 1 bis einschließlich Case 3 
(Zeile 9 - 10) initialisiert und in Case 4 (Zeile 14) auf eine 
Client-Verbindung gewartet. 
Sobald ein Client verbunden ist durchläuft die For-Schleife in Zeile 18 - 26 
insgesamt acht Buffer (jeder Task besitzt zwei Buffer). Während ein Buffer die 
zu sendenen Daten enthält, ist im zweiten Buffer ein Zustand hinterlegt, ob 
Daten (für den jeweiligen Buffer) vorhanden sind.

Sofern ein String zum Senden bereit liegt, der Buffer also ein Element an der 
Position \textit{i} enthält, ist der Wert von bufferStateEvent\{i\} in Zeile 19 
true.\\
In Zeile 20 wird das Datenpaket (String) zum Client gesendet.
Zeile 21 - 23 zeigen das Zurücksetzen des Buffers und eine Wartezeit, welche 
notwendig ist, da sonst die Sendefrequenz zu hoch ist.

Falls die Verbindung zusammenbricht oder ein ähnlicher Fehler erfolgt, der die 
Kommunikation behindert, wird in Zeile 31 - 38 die Fehlerbehandlungsroutine 
aufgerufen. Darin erfolgt der Aufruf von \textit{initSocket()}, wodurch die 
Verbindung beendet und neu initialisiert wird. Der Programmzeiger ist 
anschließend innerhalb kürzester Zeit (max. ca. 1s) wieder in Zeile 14 (warten 
auf eingehende Verbindung). In dieser Zeit ist eine Client-Verbindung nicht 
möglich.\\
Um die Verbindung dauerhaft zu prüfen muss immer gesendet werden, da so der 
Aufruf von \textit{SocketSend} einen Fehler generiert.
\label{abbCode:moduleServerComm}
\begin{lstlisting}[language=java, caption=Task ServerComm - Modul ServerComm]
PROC main()
    WHILE TRUE DO
        waitForClients;
    ENDWHILE
ENDPROC    
(...)
PROC waitForClients()
     TEST state
     CASE 1:
			(...)
     CASE 4:			
         WHILE listening DO
             IF i<=MAX_CLIENTS THEN					
                 SocketAccept server_socket,client_socket{i}\Time:=WAIT_MAX;    
					(...)
					ENDIF
         ENDWHILE
 				FOR i FROM 1 TO 25 DO
 					IF (bufferStateEvent{i}) THEN
 						SocketSend client_socket{1}\Str:=sendbufferEvent{i};
 						sendbufferEvent{i} := "";
 						bufferStateEvent{i} := false;
 						WaitTime 0.08;
 					ENDIF					
					(...)										
 				ENDFOR
     DEFAULT:
     ENDTEST   
ENDPROC    

    ERROR
        IF ERRNO=ERR_SOCK_CLOSED THEN
		(...)
        IF ERRNO=ERR_SOCK_ADDR_INUSE THEN
		(...)
        IF ERRNO=ERR_SOCK_TIMEOUT THEN
		(...)
    ENDPROC

    PROC initSocket()
        SocketClose server_socket;
        SocketClose client_socket{1};
		(...)
        listening:=TRUE;
		clientConnected:=FALSE; 
    ENDPROC
\end{lstlisting}


\subsubsection{Task EventMessages}
Der Task \textit{EventMessages} beinhaltet das Modul \textit{EventMessages} und 
ist dafür zuständig alle Informationen, Warnungen und Fehler zu senden.\\
Beispiel: Wenn der Roboter im Automatikbetrieb ein Programm abarbeitet und das 
Programm gestoppt wird erfolgt das Senden einer entsprechenden Dialog-Nachricht 
(vom Inhalt her wie auf dem Flexpendant).\\
Das Modul besteht im Grunde nur aus einem Interrupt \textit{err\_trap}, dessen 
Aufruf erfolgt sobald eine Ereignismeldung ansteht. Um den Interrupt aktiv zu 
halten wird in der Hauptroutine, nach der Definition des Interrupts, eine 
Endlos-Schleife durchlaufen (s. \ref{app:EventMessages} Zeile 1 - 7).
Im Interrupt selbst erfolgt das Setzen eines Flags mit dem das Detektierten 
eines Zyklusstop im Automatikbetrieb erfolgt. In Zeile 18 - 19 wird der zu 
sendende String an die Routine \textit{tpWriteSocket()} übergeben.\\
Die Routine \textit{tpWriteSocket()} schreibt u.A. den übergebenen String in 
den entsprechenden Buffer des Tasks, sobald ein Client verbunden ist (s. Zeile 
23 - 30).
Der String setzt sich aus folgenden Komponenten zusammen (s. Bedeutung 
\ref{tab:parameterEventMsg}):\\
msgType+robotState+::+err\_domain+::+err\_number+::+err\_type+::+str1
wodurch folgender Beispielstring vorliegen kann: :e:1::1::10::1::7
\begin{table}[H]
\caption{Bedeutung der Parameter}\label{tab:parameterEventMsg}
\renewcommand{\arraystretch}{1.5} 
\newcolumntype{C}[1]{>{\centering\arraybackslash}p{#1}}
\centering
\begin{tabular}{|p{3cm}|p{8cm}|}
\hline 
\textbf{msgType} & Nachrichtentyp \textit{e} für Event \\ 
\hline 
\textbf{robotState} & Status des Zyklus (Motoren an/aus) \\ 
\hline 
\textbf{err\_domain} & Fehlerdomäne (Auswahl der entsprechenden xml-Datei) \\ 
\hline 
\textbf{err\_number} & Fehlernummer (Auswahl des entsprechenden Stings aus der 
xml-Datei) \\ 
\hline 
\textbf{err\_type} & Fehlertyp (Info, Warnung, Fehler) \\ 
\hline 
\textbf{str1} & Erster zusätzlicher Textparameter \\ 
\hline 
\end{tabular} 
\end{table}
Der String im Code enthält mehrere X-Zeichen. Diese fungieren als Platzhalter, 
da ansonsten die Gesamtlänge des Strings (max. 80 Characters) zu hoch wäre, 
würden alle Parameter (str1 - str5) eingebettet sein. 

Eine Beschreibung der einzelnen Parameter kann in der entsprechenden 
ABB-Dokumentation nachgeschlagen werden.

\label{app:EventMessages}
\begin{lstlisting}[language=Java, caption=Task EventMessages - Modul 
EventMessages]
PROC main()
    CONNECT err_int WITH err_trap;
    IError COMMON_ERR,TYPE_ALL,err_int;
    WHILE TRUE DO
	    WaitTime 0.01;
    ENDWHILE
ENDPROC

TRAP err_trap
    IF (DOutput(DOF_CycleOn))=1 AND firstCycleStart=0 THEN
        robotState:=1;
        firstCycleStart:=1;
    ENDIF
    IF (DOutput(DOF_CycleOn)=0) AND (firstCycleStart=1) THEN
        robotState:=0;
        firstCycleStart:=0;
    ENDIF
		tpWriteSocket 
		ValToStr(robotState)+"::"+ValToStr(err_domain)+"::"+ValToStr(err_number)+"::"+ValToStr(err_type)+"::"+str1+"::"+"X"+"::"+"X"+"::"+"X"+"::"+"X",":e:";
ENDTRAP
	
PROC tpWriteSocket(string msg,string msgType)
	IF clientConnected THEN
		sendbufferEvent{cntr}:=msgType+msg+";";
		bufferStateEvent{cntr}:=TRUE;	
		cntr:=cntr+1;
		IF (cntr>=25) THEN
			cntr:=1;
		ENDIF
	ENDIF
ENDPROC
\end{lstlisting}

\subsubsection{Task MachineData}
Der Task \textit{MachineData} beinhaltet das Modul \textit{MachineData} und ist 
dafür zuständig die Maschinendaten zu senden. Diese Daten beinhalten in der 
aktuellen Version die in \ref{tab:machineData} dargestellten Parameter. Sobald 
ein Client verbunden ist, beginnt das Senden eines String-Arrays mit der 
Routine \textit{tpWriteSocket()} innerhalb einer For-Schleife (s. 
\ref{app:MachineData} Zeile 5 - 12).

\label{app:MachineData}
\begin{lstlisting}[language=Java, caption=Task MachineData - Modul 
MachineData]
PROC main()
	(...)
	isSending := TRUE;
	mData:=[machineName,projectNumber,dateOfCreation,dateOfFAT,dateOfSOP,serial,version,rtype,cid,lanip,clang,dutyTime];
    WHILE isSending DO
	    IF (clientConnected AND isSending) THEN					
	        FOR i FROM 1 TO MAX_DATA_SIZE DO
	            tpWriteSocket mData{i},":"+ValToStr(i-1)+":";
            ENDFOR                 
			isSending:=FALSE;				
        ENDIF
    ENDWHILE
	! Give other threads enough time to process
	WaitTime 5;
ENDPROC

PROC tpWriteSocket(string msg,string msgType)
	IF clientConnected THEN
		sendbufferMdata{cntr}:=msgType+msg+";";
		bufferStateMdata{cntr}:=TRUE;	
		cntr:=cntr+1;
		IF (cntr>=25) THEN
			cntr:=1;
        ENDIF
	ENDIF
ENDPROC	
\end{lstlisting}

\subsubsection{Task CycleTimer}
Der Task \textit{CycleTimer} beinhaltet das Modul \textit{CycleTimer} und wurde 
deshalb in einen externen Task implementiert, da sonst die Taktzeitaufzeichnung 
den Bewegungstask zu stark belastet (Latenzen).\\
Das Modul besteht vom Prinzip her aus den Interrupts \textit{nextCTtrap} und 
\textit{resetCTtrap} (s. \ref{app:CycleTimer} Zeile 8 - 14). Um diese aktiv zu 
halten wird in der Hauptroutine eine Endlos-Schleife durchlaufen (s. Zeile 1 - 
6). Durch den Aufruf des Interrupts \textit{nextCTtrap} erfolgt der 
Routinen-Aufruf von \textit{NextCycleTime()}, wodurch die entsprechenden Daten 
an den Socket gesendet werden (s. Zeile 18 - 21). In Zeile 18 - 19 wird der 
Routine \textit{tpWriteSocket()} als msgType :l: übergeben. Hierdurch wird der 
übergebene String geloggt. Zeile 20 und 21 sendet die aktuelle Zykluszeit und 
dessen Durchschhnitt (msgType :c1: und :c2:).

\label{app:CycleTimer}
\begin{lstlisting}[language=Java, caption=Task CycleTimer - Modul 
CycleTimer]
PROC main()
	(...)
	WHILE TRUE DO		
		WaitTime 0.01;
	ENDWHILE
ENDPROC
	
TRAP nextCTtrap
	NextCycleTime;	
ENDTRAP
	
TRAP resetCTtrap
	ResetCycleTimes;	
ENDTRAP
	
PROC NextCycleTime()		
	(...)
    tpWriteSocket "Gesamtzahl der gefertigten Teile:: 
    "+ValToStr(nCyclesShow),":l:";
	tpWriteSocket ValToStr(nCycleTime),":c1:";
	tpWriteSocket ValToStr(cycleTimeMean{1}),":c2:";
	(...)
ENDPROC
	
(...)
	
PROC tpWriteSocket(string msg,string msgType)
	IF clientConnected THEN
		sendbufferCycleTime{cntr}:=msgType+msg+";";
		bufferStateCycleTime{cntr}:=TRUE;	
		cntr:=cntr+1;
		IF (cntr>=25) THEN
			cntr:=1;
        ENDIF
	ENDIF
ENDPROC
\end{lstlisting}