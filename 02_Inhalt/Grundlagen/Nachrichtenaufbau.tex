\section{Nachrichtenaufbau}
Jede Nachricht, die von der ABB-Steuerung gesendet und von der \gls{app} 
korrekt 
empfangen werden soll, muss folgendes Format aufweisen:

Nachricht = :\{command\}:\{message\};

Bei der Standardkommunikation sind für \{command\} folgende Parameter bereits 
fest vergeben:

\{0...n-1\}: Machine data mit n aus $\mathbb{N}$ \\
Dieser Parameter kann erweitert werden. Hierbei muss die App (Android), als 
auch das Modul (ABB) angeasst werden.

\{l\}: Logging \\

\{c1\}: Cycle time (Aktueller Wert)\\
\{c2\}: Cycle time (Durchschnittswert)\\
Diese Parameter können erweitert werden mit z.B. \{c3\}, \{c4\}, etc.

\{e\}: Events\\

\{p\}: Ping\\
Dieser Parameter wird zur Überprüfung verwendet, ob der Empfänger noch 
vorhanden ist. Dies wird in \ref{sec:ChapterABB} erläutert.

Sofern in einer der Nachrichtenkomponenten (\{command\} oder \{message\}) ein 
Doppelpunkt enthalten ist, so muss dieses mit einem führenden Doppelpunkt 
angegeben werden. Ansonsten wird es nicht als, im Satz enthaltenes, Zeichen 
erkannt. Die Nachricht wird dadurch fehlerhaft dargestellt.

Nachrichtenbeispiele bei der Standardkommunikation sind folgend aufgeführt:

\textbf{Machine data}\\
:0:Multiflex; \\
:1:20184; \\
:2:2015; \\
:3:10.10.15; \\

Die Maschinendaten beinhalten die wichtigsten Informationen über das System und 
werden als \{command\}, beginnend bei 0, durchnummeriert.\\
Aktuell werden die in \ref{tab:machineData} dargestellten Parameter ausgegeben. 
Die in der rechten Spalte aufgezeigten Werte sind als Beispiel anzunehmen.
% Tabelle für Parameter Machine Data
\begin{table}
\caption{Parameter der Maschinendaten}\label{tab:machineData}
\renewcommand{\arraystretch}{1.5} 
\newcolumntype{C}[1]{>{\centering\arraybackslash}p{#1}}
\centering
\begin{tabular}{|p{5cm}|p{5cm}|}
\hline 
\textbf{Maschinentyp} & Maxiflex \\ 
\hline 
\textbf{Projektnummer} & 20184 \\ 
\hline 
\textbf{Herstelljahr} & 2015 \\ 
\hline 
\textbf{Werksabnahme} & 10.10.15 \\ 
\hline 
\textbf{Produktionsstart} & 15.10.15 \\ 
\hline 
\textbf{Serien-Nr.} & 20184 \\ 
\hline 
\textbf{Software-Version} & 20184 \\ 
\hline 
\textbf{Robotertyp} & 20184 \\ 
\hline 
\textbf{Steuerungs-ID} & 20184 \\ 
\hline 
\textbf{IP-Adresse} & 192.168.1.2 \\ 
\hline 
\textbf{Sprache} & DE \\ 
\hline 
\textbf{Betriebsstunden} & 10.5 \\ 
\hline 
\end{tabular} 
\end{table}
\\[20pt]
\textbf{Logging}\\
:l:Gesamtzahl der gefertigten Teile 81; \\

\textbf{Cycle time}\\
:c1:2.545; \\
:c2:2.273; \\

\textbf{Events}\\
:e:1::1::10::1::X::X::X::X; \\
:e:1::1::17::1::T\_ROB1::X::X::X; \\
:e:3::28::10::1::63::7::X::X; \\




