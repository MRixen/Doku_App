\newglossaryentry{floribot}
{
  name=FloriBot,
  description={Roboter mit Differentialantrieb der Hochschule Heilbronn}
}

\newglossaryentry{floribothmi}
{
  name=FloriBot HMI,
  description={App zum Steuern des Roboters FloriBot}
}

\newglossaryentry{turtlesim}
{
  name=turtlesim,
  description={Paket von ROS zum Lehren verschiedener ROS-Funktionen}
}

\newglossaryentry{topic}
{
  name=Topic,
  description={Kommunikationsleitung zwischen Nodes}
}

\newglossaryentry{node}
{
  name=Node,
  description={Executables von ROS}
}

\newglossaryentry{executables}
{
  name=Executables,
  description={Ausführbares Programm}
}

\newglossaryentry{dumbphone}
{
  name=Dumbphone,
  description={Ein mobiles Telefon, dass über keinen Internet-Anschluss und Touchscreen verfügt}
}

\newglossaryentry{hierarchieviewer}
{
  name=HierarchieViewer,
  description={Programm zum Analysieren der Performance eines grafischen Layouts einer Android-Anwendung}
}

\newglossaryentry{ninepatchdraw}
{
  name=9PatchDraw,
  description={Programm zum Erstellen von skalierbaren Hintergrundgrafiken einer Android-Anwendung}
}

\newglossaryentry{aistarter}
{
  name=aiStarter,
  description={Programm zum Bereitstellen von fundamentalen Funktionen für MIT App Inventor}
}

\newglossaryentry{appinventor}
{
  name=MIT App Inventor,
  description={Programm zum Erstellen von Android-Anwendungen in einem Webbrowser}
}

\newglossaryentry{ai2companion}
{
  name=MIT AI2 Companion,
  description={App stellt Funktionen bereit, um eine Android-Anwendung mit MIT App Inventor zu entwickeln}
}

\newglossaryentry{fragment}
{
  name=Fragment,
  description={Ermöglicht das Laden einer lokalen Benutzerschnittstelle in einer Activity}
}

\newglossaryentry{activity}
{
  name=activity,
  description={Ermöglicht das Laden einer grafischen Benutzeroberfläche einer Android Anwendung},
  plural={activities}
}

\newglossaryentry{pubsubtutorial}
{
  name=PubSubTutorial,
  description={App zum Testen der Kommunikation zwischen ROS und Android}
}

\newglossaryentry{rosteleop}
{
  name=ROS-Teleop,
  description={App zum Steuern eines Roboters mit ROS}
}

\newglossaryentry{rosjava}
{
  name=ROSJava,
  description={Paket, das u.a. client libraries für die Kommunikation in ROS beinhaltet}
}

\newglossaryentry{actionbar}
{
  name=action bar,
  description={Obere Leiste der HMI}
}

\newglossaryentry{funktionsleiste}
{
  name=Funktionsleiste,
  description={Untere Leiste der HMI}
}

\newglossaryentry{betriebsmoditasten}
{
  name=Betriebsmodi-Tasten,
  description={Zum Aktivieren der verschiedenen Betriebsmodi, einschließlich der Sensorsteuerung}
}

\newglossaryentry{sensorsteuerung}
{
  name=Sensorsteuerung,
  description={Steuerungsart zum Verfahren des Roboters mithilfe des Beschleunigungssensors}
}

\newglossaryentry{sensormanage}
{
  name=SensorManage,
  description={App zum Auslesen und Modifizieren der Beschleunigungswerte}
}

\newglossaryentry{roscomm}
{
  name=ROSComm,
  description={App zum Testen der Kommunikation zwischen ROS und Android}
}

\newglossaryentry{publisher}
{
  name=Publisher,
  description={Dient dem Senden von Daten von der HMI zum Roboter}
}

\newglossaryentry{subscriber}
{
  name=Subscriber,
  description={Dient dem Empfangen von Daten vom Roboter}
}


\newacronym{hmi}{HMI}{Human Machine Interface}
\newacronym{app}{App}{Application}
\newacronym{fre}{FRE}{Field Robot Event}
\newacronym{adb}{ADB}{Android Debug Bridge}
\newacronym{ide}{IDE}{Interated Development Environment}
\newacronym{sdk}{SDK}{Software Development Kit}
\newacronym{jdk}{JDK}{Java Development Kit}
\newacronym{jre}{JRE}{Java Runtime Environment}
\newacronym{ros}{ROS}{Robot Operating System}
\newacronym{gui}{GUI}{Graphical User Interface}
\newacronym{oha}{OHA}{Open Handset Alliance}
\newacronym{dvm}{DVM}{Dalvik Virtual Machine}
\newacronym{idc}{IDC}{International Data Corporation}
\newacronym{avd}{AVD}{Android Virtual Device}
\newacronym{api}{API}{Application Programming Interface}
\newacronym{iso}{ISO}{International Organization for Standardization}
\newacronym{uri}{URI}{Unified Ressource Identifier}
\newacronym{adt}{ADT}{Android Developer Tools}
\newacronym{ui}{UI}{User Interface}
\newacronym{id}{ID}{Identification}
\newacronym{bsd}{BSD}{Berkeley Software Distribution}
\newacronym{ks}{KS}{Koordinatensystem}